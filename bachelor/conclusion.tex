\chapter{Заключение}
\section{Достигнутые результаты}
В процессе работы над дипломным проектом:
\begin{itemize}
  \item Изучено программное обеспечение конкурса КИО.
  \item Проанализированы достоинства и недостатки различных платформ, которые можно было использовать для реализации сервера. В итоге выбран Node.js и обосновано его использование;
  \item Спроектирована общая структура сервера, выбраны протоколы обмена между частями системы;
  \item Для вызова функций из динамической библиотеки используется модуль node-ffi;
  \item Описана процедура установки и настройки node-ffi под Linux;
  \item Под Windows установить этот модуль не удалось. Попытки описаны в ~\ref{windowsInstall}. 
  \item Реализован сервер на Javascript, который принимает запросы, вызывает нужные функции и возвращает результат;
  \item Реализован пример на Flash, который демонстрирует соединение приложения с динамической библиотекой через созданный сервер.
  \item Код протестирован и отвечает поставленном в техническом задании требованиям. Код прокомментирован и документирован в этой пояснительной записке.
  \item Создан Shell Script для установки node.js для Ubuntu. При сборке всего установщика конкурса при выполнении данного скрипта будет установлена вся серверная часть. 
\end{itemize}
\section{Дальнейшее развитие проекта}
\begin{itemize}
  \item Реализация сервера под Windows (создание скомпилированных версий node-ffi под каждую версию Windows);
  \item Обработка ошибок при создании запроса программистом, создающим Flash-приложение для (удобства отладки программистом).
  \item Создание утилиты, которая извлекает сведения о функциях из динамической библиотеки и приводит данные к формату base.txt.
\end{itemize}
