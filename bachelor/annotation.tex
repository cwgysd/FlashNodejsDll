\begin{center}
\textbf{Аннотация}
\end{center}
В данной дипломной работе разработан программный продукт, представляющий собой web-сервер на базе Node.js, который позволяет вызывать функции, реализованнные в динамической библиотеке из Flash-приложений или других технологий (например, Javascript, Silverlight) для создания клиентской части web-приложения.\\
В дипломной работе разработана общая схема системы, проанализированы достоинства и недостатки возможных средств реализации, и выбрано наилучшее с точки зрения сформулированных критериев. \\
С целью сравнения приведены листинги web-серверов, реализованных на разных языках (C, Python, PHP, Node.js).\\
Разработан исходный код сервера на Node.js, описан процесс настройки Node.js под Linux.\\
Эту работу можно использовать для дальнейшей разработки задач КИО на Flash, требующих использование функций из динамических библиотек. \\
В заключении описаны возможные перспективы программного продукта. Большая часть поставленных в техническом задании целей была достигнута.

\newpage
\begin{center}
\textbf{Abstract}
\end{center}
In this diploma has been created a software, which is a server made on Node.js, which allows to connect dynamic libraries with Flash-applications or other techono-logies (ex. Javascript, SilverLight) for making client parts of web-services.\\
The common structure of the client has been shown, advantages and disadvantages of possible platforms has been analyzed, and the best solution according to formula-ted criteria has been chosen.\\
In order to compare listings of web-servers, made on C, Python, PHP, Node.js has been demonstrated.\\
The project has been made on Node.js. Process of setting-up Node.js on Linux has been shown.\\
This project can be useful for the future development of the competition CEO (to Construct, to Explore, To optimize) on Flash.\\
In the end there are future possibilities of this product. Allmost all of the goals, described in specification, has been reached.
\newpage
\begin{center}
\textbf{Ключевые слова}
\end{center}
\begin{itemize}
  \item Node.JS
  \item Javascript
  \item Создание, вызов динамической библиотеки
  \item Flash
  \item ActionScript
  \item Flash player debugger
  \item npm
  \item node-ffi
\end{itemize}